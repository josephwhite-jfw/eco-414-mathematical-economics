% Preamble
\documentclass[11pt, letterpaper]{article}
\usepackage{newtxtext}
\usepackage{newtxmath}
\title{Reflection Paper -- Mathematical Economics}
\author{Joe White}
\date{\today}

\begin{document}
\maketitle

% Start Section 1: Covers the motivation as well as background of the paper
\section{Motivation and Background}
\subsection{Motivation}
The motivation behind this paper is to understand how ownership structure, geography, and prices are all linked together. It was very important to understand this relationship as market power, structure, and mergers are all important towards creating competitive environments. The research question of this paper is how do mergers in differentiated industries differ in their effect on prices. There are two levels to this paper: Creating a structural estimation of demand and supply that takes into account market geography, and understanding how mergers affect prices in these differentiated industries. From a policy perspective, antitrust authorities such as the Federal Trade Commission (FTC) examine mergers to understand how these mergers affect consumer prices. It is important for these authorities to set policy so that competition is upheld in certain industries so that no one individual or firm holds complete market power, which makes consumers worse off. The positioning is the literature is such that before this paper, previous research had looked at location effects alone, or the multi-store structure in general, but no one had examined both in this structural pricing estimation context. 

\subsection{Background}

In his paper \textit{"The Effect of Ownership Structure on Prices in Geographically Differentiated
Industries"} Thomadsen looks at mergers and franchisees in Santa Clara California. The data he uses in his paper are the prices, locations, and characteristics of Burger King and McDonald's restaurants in this region. The choice of only looking at these two fast food chains stems from multiple different reasons. The first being that these two companies are two of the largest fast food chains in the United States, and in his study area they make up a decent sized portion of annual consumption. The second reason is that most of the outlets in the United States are owned and operated by franchisees. These franchisees purchase the right to use Burger King or McDonald's established brand, business model, and operating system in exchange for a fee and royalties. The third reason for using these two businesses is that each have very consistent product offerings across each individual store and franchise, unlike many other retail outlets. The data used in this paper comes directly from Thomadsen himself. He takes the place of the consumer and physically visited the restaurants in the region, either taking a picture of the menu and prices, or writing it down. Not every single Burger King or Mcdonald's should have been included in the sample, and his criteria for excluding outlets is as follows. Outlets owned by the corporation, and outlets in the nearby airport or military base. In order to simplify his study, he also only incorporates the menu prices of the value meals for the Big Mac for McDonald's and the Whopper for Burger King, as they are the most purchased menu items for each corporation.

% Start Section 2: Outlines the paper and explains what the paper does, the papers models, and results
\section{Overview}
\subsection{What does the paper accomplish?}
\subsection{Methodology}
\subsection{Results}

\section{Model}

\subsection{Motivation}

The motivation behind incorporating this structural model in his paper, just like for any structural model, is that regressions only give a small explanation for what is happening. Estimating effects using structural models is "alternative to reflect the complexity and multidimensional present in certain theoretical discussions, providing more accurate and reliable results than usual techniques, such as multiple linear regression" (make sure to cite here). Regressions cannot account for very much of anything, and therefore they can be somewhat biased when they try to explain causation instead of correlation. They also do not explain what is actually happening under the hood, ie they cannot explain the theory behind the economic forces that actually influence prices. 

\subsection{Demand}

Conditional indirect utility for consumer $i$ at location $j$ is:

\begin{equation}
V_{i,j} = X^{\prime}\beta - D_{i,j}\delta - P_{j}\gamma + \eta_{i,j}
\end{equation}


Where $X^{\prime}$ is a vector indicating chain index, if the location has a drive through or play area, and if it is located in a mall. $D_{i,j}$ represents the distance between consumer $i$ and outlet $j$. The price of a meal at outlet $j$ is denoted by $P_j$.

If a consumer consumes elsewhere (other than BK or McDonald's) they have conditional utility:

\begin{equation}
V_{i,0} = \beta_{0} + M_{i}\pi + \eta_{i,0}
\end{equation}

\begin{equation}
S_{i,b}(P, X, M \mid \beta, \delta, \gamma, \pi) = \int_{A_j} f(\eta_{i})\,d\eta_{i}
\end{equation}

\begin{equation}
A_j = \{\eta_i \mid (V_{i,j} > V_{i,t} \;\forall t \neq j) \cap (V_{i,j} > V_{i,0})\}
\end{equation}

\begin{equation}
S_{j,b}(P, X, M \mid \beta, \delta, \gamma, \pi)
=
\frac{
    e^{X'_j \beta - D_{b,j} \delta - P_j \gamma}
}{
    e^{\pi M} + \sum_{t=1}^{J} e^{X'_t \beta - D_{b,t} \delta - P_t \gamma}
}.
\end{equation}

\begin{equation}
Q_{j}(P,X \mid \beta, \delta, \gamma, \pi) = \sum_{b} \sum_{M} h(b,M)S_{i,b}(P, X, M \mid \beta, \delta, \gamma, \pi)
\end{equation}

\begin{equation}
\frac{\partial Q_{j}(P,X \mid \beta, \delta, \gamma, \pi)}{\partial P_k} = \frac{\partial \sum_{b} \sum_{M} h(b,M)S_{i,b}(P, X, M \mid \beta, \delta, \gamma, \pi)}{\partial P_k}
\end{equation}

\subsection{Supply}

\begin{equation}
\Pi_f = \sum_{j \in F_f} (r_{k} P_{j} Q_{j}(P) - c_{j} Q_{j}(P) - FC_j)
\end{equation}

\begin{equation}
\Pi_f = \sum_{j \in F_f}(P_j Q_{j}(P) - (\frac{c_j}{r_k})Q_{j}(P) - \frac{FC_j}{r_k})
\end{equation}

\begin{equation}
C_j = (C_k + \varepsilon_j)
\end{equation}

\begin{equation}
Q_{j}(P) = \sum_{r \in F_f} (P_{r} - C_{k} - \varepsilon_{r}) \frac{\partial Q_{r}(P)}{\partial P_j} = 0
\end{equation}

\begin{equation}
\Omega_{j,r} = 
\begin{cases}
\displaystyle \frac{\partial Q_r}{\partial P_j}, & \text{if $r$ and $j$ have the same owner}, \\[8pt]
0, & \text{otherwise}.
\end{cases}
\end{equation}

\begin{equation}
Q(P) + \Omega(P - C - \varepsilon) = 0
\end{equation}

\begin{equation}
Q(P, X \mid \theta) + \Omega(P, X \mid \theta)(P - C - \varepsilon) = 0
\end{equation}

\begin{equation}
\varepsilon = P - C + \Omega(P, X \mid \theta)^{-1} Q(P, X \mid \theta)
\end{equation}

\begin{equation}
E[\varepsilon_{j}(\theta^{*}) \mid Z_j] = 0
\end{equation}

\begin{subequations}
\label{}

\begin{align}
G_J(\theta) 
&= \frac{1}{J} \sum_{j=1}^{J} Z_j \, \varepsilon_j(\theta)
\label{}
\\[8pt]
G_J(\theta) 
&= \frac{1}{J} \sum_{j=1}^{J}
    \left[
        R_{18\text{-}29}
        \frac{Q_j(M_{\text{Age}},\theta)}{\text{Pop}(M_{\text{Age}})}
        \;-\;
        R_{\text{Age}}
        \frac{Q_j(M_{18\text{-}29},\theta)}{\text{Pop}(M_{18\text{-}29})}
    \right]
\label{}
\\[8pt]
G_J(\theta) 
&= \frac{1}{J} \sum_{j=1}^{J}
    \left[
        R_{\text{Male}}
        \frac{Q_j(M_{\text{Female}},\theta)}{\text{Pop}(M_{\text{Female}})}
        \;-\;
        R_{\text{Female}}
        \frac{Q_j(M_{\text{Male}},\theta)}{\text{Pop}(M_{\text{Male}})}
    \right]
\label{}
\\[8pt]
G_J(\theta) 
&= \frac{1}{J} \sum_{j=1}^{J}
    \left[
        R_{\text{Black}}
        \frac{Q_j(M_{\text{White}},\theta)}{\text{Pop}(M_{\text{White}})}
        \;-\;
        R_{\text{White}}
        \frac{Q_j(M_{\text{Black}},\theta)}{\text{Pop}(M_{\text{Black}})}
    \right]
\label{}
\end{align}

\end{subequations}

\begin{thebibliography}{9}
\bibitem{article}
Miriane Lucindo Zucoloto \emph{Structural Equation Modelling as an alternative to Multiple Linear Regression Models},.
\end{thebibliography}

\end{document}
